% source: https://www.jianshu.com/p/28552706e811

%==============================常用宏包、环境==============================%
\documentclass[a4]{article}
\usepackage{ctex} % For Chinese characters
\usepackage{amsmath, amsthm}
\usepackage{minted}
\usepackage{geometry} % 设置页边距
\usepackage{fontspec}
\usepackage{graphicx}
\usepackage{fancyhdr} % 自定义页眉页脚
\usepackage{longtable}
\setsansfont{Consolas} % 设置英文字体
\setmonofont[Mapping={}]{Consolas} % 英文引号之类的正常显示,相当于设置英文字体
\geometry{left=1cm,right=1cm,top=2cm,bottom=0.5cm} % 页边距
\setlength{\columnsep}{30pt}
% \setlength\columnseprule{0.4pt} % 分割线
%==============================常用宏包、环境==============================%

%==============================页眉、页脚、代码格式设置==============================%
% 页眉、页脚设置
\pagestyle{fancy}
% \lhead{CUMTB}
\lhead{zqy1018}
\chead{}
% \rhead{Page \thepage}
\rhead{第 \thepage 页}
\lfoot{} 
\cfoot{}
\rfoot{}
\renewcommand{\headrulewidth}{0.4pt} 
\renewcommand{\footrulewidth}{0.4pt}

\theoremstyle{definition}
\newtheorem*{theorem}{Theorem}

\newcommand{\myminted}[2]{
	\inputminted[
	breaklines,
	breakafter=()\&|,
	fontsize=\small,
	frame=single,
	framesep=1mm,
	framerule=1pt,
	linenos,
	numbersep=1pt,
	tabsize=4,
	]{#1}{#2}
}

\usepackage[pdfstartview=FitH,
CJKbookmarks=true,
bookmarksnumbered=true,
bookmarksopen=true,
colorlinks,
pdfborder=001,
linkcolor=blue,
anchorcolor=blue,
citecolor=blue,
]{hyperref}
\hypersetup{hidelinks}

\usepackage{enumitem}
\setenumerate[1]{itemsep=0pt,partopsep=0pt,parsep=\parskip,topsep=5pt}
\setitemize[1]{itemsep=0pt,partopsep=0pt,parsep=\parskip,topsep=5pt}
\setdescription{itemsep=0pt,partopsep=0pt,parsep=\parskip,topsep=5pt}


%==============================页眉、页脚、代码格式设置==============================%

%==============================标题和目录==============================%

\begin{document}%\small
	
\title{个人常用代码库}
\author{zqy1018}
\renewcommand{\today}{\number\year 年 \number\month 月 \number\day 日}

\maketitle
%\begin{titlepage}
%\maketitle
%\end{titlepage}

\pagestyle{empty}
\renewcommand{\contentsname}{目录}
\tableofcontents
\newpage\clearpage
\newpage
\pagestyle{fancy}
\setcounter{page}{1}   %new page
%==============================标题和目录==============================%

%==============================正文部分==============================%

%------------------------------数据结构------------------------------%
\section{数据结构}

%\subsection{二维 Sparse Table}


%\subsection{带位置修改堆}

\subsection{非旋转 Treap}

\myminted{cpp}{data_structure/fhqtreap_full.cpp}


%对每一个位置维护位置 $pos$ 和插入时间 $id$ 即可。


%\newpage

%------------------------------图论------------------------------%




%------------------------------数学------------------------------%

\section{数学}

\subsection{快速乘}

\myminted{cpp}{math/fstmul_full.cpp}

\subsection{扩展 Euclid 及应用}

\myminted{cpp}{math/ex_euclid_full.cpp}

\subsection{线性同余不等式}

给定 $0 \le l \le r < m, d < m$,找 $l \le dx \bmod m \le r$ 的最小非负整数解。详见 POJ 3530。

\myminted{cpp}{math/modinq_full.cpp}

\subsection{求逆元}

\myminted{cpp}{math/inverse_full.cpp}

\subsection{值域固定的快速 $\gcd$}

\myminted{cpp}{math/fstgcd_full.cpp}

\subsection{CRT 及其扩展}

普通 CRT:对于 $n$ 个形如 $x \equiv a_i \pmod {m_i}$ 的线性方程,要求 $m_i$ 间两两互质。设 $M = \prod m_i, M_i = \frac{M}{m_i}, b_i M_i \equiv 1 \pmod {m_i}$,那么 $x  = \sum a_i b_i M_i$ 是唯一解。

扩展 CRT:考虑每次合并两个线性方程:$x\equiv b_1 \pmod {m_1} , ax\equiv b_2 \pmod {m_2}$。写 $x=b_1+tm_1$,带入得 $m_1 a t\equiv b_2 - ab_1 \pmod {m_2}$。解出 $t$,如 $t\equiv c \pmod {m_2}$,则 $t=c+dm_2$,带入,从而 $x\equiv b_1+cm_1 \pmod {\operatorname{lcm}(m_1, m_2)}$。

\myminted{cpp}{math/CRT_full.cpp}

\subsection{组合数取模}

普通 Lucas:$p$ 为质数时,等价于把 $n, m$ 在 $p$ 进制下分解后求组合数再乘起来。
$$
\binom{n}{m} \equiv \binom{\lfloor \frac{n}{p} \rfloor}{\lfloor \frac{m}{p} \rfloor} \binom{n \bmod p}{m \bmod p} \pmod p
$$

扩展 Lucas:$p$ 不为质数时将其分解成 $p_i^k$,对每一个分别求解然后用 CRT 合并。分开处理,对阶乘中 $p$ 的倍数计算 $p$ 的幂次,对其他部分找循环节。

\myminted{cpp}{math/Lucas_full.cpp}

\subsection{高斯消元}

\myminted{cpp}{math/gauss_elimination_AA.cpp}
%\subsection{线性基}

%自高到低。

\subsection{大数分解}

\myminted{cpp}{math/factorization_full.cpp}

%\subsection{BSGS 及其扩展}






%\subsection{类 Euclid 算法}

%\subsection{找原根}


\subsection{博弈论}

\input{math/game_theory.tex}

\subsection{数学知识}

%\subsubsection{各种常数}

%\begin{itemize}
	%\item Ramsey 数:
%\end{itemize}

\subsubsection{循环矩阵乘法}

定义:方阵,下一行为上一行的右循环移位。关于加法、乘法封闭。乘法规则:
$$
c_{k} = \sum_{(i + j) \bmod n = k} a_i b_j
$$

可以只用第一行表示。用行向量乘方便,$aB=c \iff AB=C$。

%\subsubsection{Doob 鞅停时定理}


\subsubsection{组合学定理}

\begin{itemize}
	%\item (Raney) 
	\item (Ramsey) $\forall p \ge r(m, n), K_p \rightarrow K_m, K_n$. 
	\item (Dilworth) 最小链覆盖的链数等于最长反链长度,反之亦然。 
\end{itemize}



%------------------------------杂项------------------------------%

\section{杂项}

\subsection{Zeller 公式}

输入年月日,告诉是周几,返回 1 就是星期一,7 就是星期天。

\myminted{cpp}{misc/Zeller_full.cpp}

\subsection{自适应 Simpson 积分}

\myminted{cpp}{misc/asr_full.cpp}

\subsection{位运算}

\myminted{cpp}{misc/bit_operation_full.cpp}

\subsection{Python 读到 EOF}

\myminted{python}{misc/python_EOF_full.py}

%\subsection{积分表和微分方程}




%==============================正文部分==============================%
\end{document}